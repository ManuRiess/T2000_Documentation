% Präambel
\documentclass[%
12pt,							% Schriftgröße
a4paper,						% Papierformat
oneside, 						% einseitiges (oneside) oder zweiseitiges (twoside) Dokument
listof=totoc, 					% Tabellen- und Abbildungsverzeichnis ins Inhaltsverzeichnis
bibliography=totoc,				% Literaturverzeichnis ins Inhaltsverzeichnis aufnehmen
titlepage, 						% Titlepage-Umgebung statt \maketitle
headsepline, 					% horizontale Linie unter Kolumnentitel
%abstracton,					% Überschrift beim Abstract einschalten, Abstract muss dazu in {abstract}-Umgebung stehen
DIV=12,							% Satzspiegeleinstellung, 12 ist Standar bei KOMA
%BCOR=6mm,						% Bindekorrektur, die den Seitenspiegel um 6mm nach rechts verschiebt,
cleardoublepage=empty,			% Stil einer leeren eingefügten Seite bei Kapitelwechsel
parskip,							% Absatzabstand bei Absatzwechsel einfügen
ngerman
]{scrbook}			
\usepackage{scrhack}
\usepackage[utf8]{inputenc} 	% ermöglicht die direkte Eingabe von Umlauten
\usepackage[T1]{fontenc} 		% Ausgabe aller zeichen in einer T1-Codierung (wichtig für die Ausgabe von Umlauten!)
\usepackage{babel} 	% deutsche Trennungsregeln und Übersetzung der festcodierten Überschriften
\setlength{\parindent}{0ex} 	% bei neuem Abschnitt nicht einrücken
%------
% Folgende Einstellungen entsprechen den Vorgaben der Leitlinien
\usepackage[onehalfspacing]{setspace}
% Ende Leitlinien
%------
%------
% Folgende Einstellungen sind bei größeren Arbeiten mit viel Text zu empfehlen.
% Hierbei oben DIV=16 einstellen und Zeile \usepackage[onehalfspacing]{setspace} auskommentieren.
%\linespread{1.2}\selectfont     % Zeilenabstand erhöhen - größere Werte als 1.2 nicht verwenden!!
% Ende Einstellung große Arbeiten mit viel Text.
%------

\usepackage{siunitx}			% Vereinfachte Eingabe von Einheiten in Formeln
\sisetup{
	number-unit-product = \;,
	inter-unit-product = \:,
	exponent-product = \cdot,
	output-decimal-marker = {,}
}

\usepackage{graphicx}  			% Einbinden von Grafiken erlauben
\usepackage[format=hang,		% Formatierungen von Unter- / Überschriften
font=normal,
labelfont=bf,
justification=RaggedRight,
singlelinecheck=true,
aboveskip=1mm
]{caption}

\usepackage[backend=biber, %% Hilfsprogramm "biber" beim Compilieren nutzen (statt "biblatex" oder "bibtex")
style=alphabetic, %% Zitierstil (siehe Dokumentation)
natbib=true, %% Bereitstellen von natbib-kompatiblen Zitierkommandos
hyperref=true, %% hyperref-Paket verwenden, um Links zu erstellen
]{biblatex}
\addbibresource{literature/literatur1.bib} %% Einbinden der bib-Datei. Endung .bib unbedingt ergänzen
\addbibresource{literature/literatur2.bib} %% Einbinden mehrerer bib-Dateien mit zusätzlichem \addbibresource - Befehl

% Folgende Zeilen sind auszukommentieren, falls runde Klammern und ein vgl. bei Zitaten erscheinen sollen.
%\makeatletter
%\renewcommand{\@cite}[2]{(vgl. {#1\if@tempswa , #2\fi})} 
%\renewcommand{\@biblabel}[1]{(#1)}
%\makeatother

\usepackage{pdfpages}

\usepackage{enumitem}			% Erlaubt Änderung der Nummerierung in der Umgebung enumerate

\usepackage{amsmath}			% Ergänzungen für Formeln
\usepackage{textcomp} 			% zum Einsatz von Eurozeichen u. a. Symbolen
\usepackage{eurosym}			% bessere Darstellung Euro-Symbol mit \euro
\newcommand*\diff{\mathop{}\!\mathrm{d}}	% Differentialzeichen
\newcommand*\Diff[1]{\mathop{}\!\mathrm{d^#1}} % Differentialzeichen höherer Ableitung
\newcommand*\jj{\mathop{}\!\mathrm{j}}	% Komplexe Zahl j

\usepackage[					% Einstellunge Paket hyperref
hyperfootnotes=false,			% im pfd-Output Fußnoten nicht verlinken
hidelinks
]{hyperref}

\usepackage{makeidx}			% Paket zur Erstellung eines Index
\usepackage[intoc]{nomencl} 	% zur Erstellung des Abkürzungsberzeichnisses

\usepackage[					% Einstellungen für Fußnoten
bottom,							% Ausrichtung unten
multiple,						% Trennung durch Seperator bei mehreren Fußnoten
hang,
marginal
]{footmisc}

\usepackage{calc}				% Paket zum Berechnen von Längen z.B. 0.8\linewidth

\usepackage{xcolor} 			% einfache Verwendung von Farben in nahezu allen Farbmodellen

\usepackage{listings}			% Darstellung von Quellcode mit den Umgebungen {lstlisting}, \lstinline und \lstinputlisting
\lstset{literate=				% Damit können Umlaute innerhalb Listings geschrieben werden
	{Ö}{{\"O}}1
	{Ä}{{\"A}}1
	{Ü}{{\"U}}1
	{ß}{{\ss}}1
	{ü}{{\"u}}1
	{ä}{{\"a}}1
	{ö}{{\"o}}1
}
\definecolor{mygreen}{rgb}{0,0.6,0}
\definecolor{mygray}{rgb}{0.5,0.5,0.5}
\definecolor{mymauve}{rgb}{0.58,0,0.82}
\lstset{ %
	backgroundcolor=\color{white},   % choose the background color; you must add \usepackage{color} or \usepackage{xcolor}; should come as last argument
	basicstyle=\footnotesize,        % the size of the fonts that are used for the code
	breakatwhitespace=false,         % sets if automatic breaks should only happen at whitespace
	breaklines=true,                 % sets automatic line breaking
	captionpos=t,                    % sets the caption-position to (b) bottom or (t) top
	commentstyle=\color{mygreen},    % comment style
	deletekeywords={...},            % if you want to delete keywords from the given language
	escapeinside={\%*}{*)},          % if you want to add LaTeX within your code
	escapeinside={(*@}{@*)},
	extendedchars=true,              % lets you use non-ASCII characters; for 8-bits encodings only, does not work with UTF-8
	frame=none,	                   	% "single" adds a frame around the code; "none"
	keepspaces=true,                 % keeps spaces in text, useful for keeping indentation of code (possibly needs columns=flexible)
	keywordstyle=\color{blue},       % keyword style
	language=[LaTeX]TeX,             % the language of the code
	morekeywords={*,nomenclature},   % if you want to add more keywords to the set
	numbers=left,                    % where to put the line-numbers; possible values are (none, left, right)
	numbersep=5pt,                   % how far the line-numbers are from the code
	numberstyle=\tiny\color{mygray}, % the style that is used for the line-numbers
	rulecolor=\color{black},         % if not set, the frame-color may be changed on line-breaks within not-black text (e.g. comments (green here))
	showspaces=false,                % show spaces everywhere adding particular underscores; it overrides 'showstringspaces'
	showstringspaces=false,          % underline spaces within strings only
	showtabs=false,                  % show tabs within strings adding particular underscores
	stepnumber=1,                    % the step between two line-numbers. If it's 1, each line will be numbered
	stringstyle=\color{mymauve},     % string literal style
	tabsize=2,	                   % sets default tabsize to 2 spaces
	title=\lstname                   % show the filename of files included with \lstinputlisting; also try caption instead of title
}

\makeindex						% Indexverzeichnis erstellen
\makenomenclature				% Abkürzungsverzeichnis erstellen

% -----------------------------------------------------------------------------------------------------------------
% Zum Aktualisieren des Abkürzungsverzeichnisses (Nomenklatur) bitte auf der Kommandozeile folgenden Befehl aufrufen :
% makeindex <Dateiname>.nlo -s nomencl.ist -o <Dateiname>.nls
% Oder besser: Kann in TexStudio unter Tools-Benutzer als Shortlink angelegt werden
% Konfiguration unter: Optionen-Erzeugen-Benutzerbefehle: makeindex -s nomencl.ist -t %.nlg -o %.nls %.nlo
% -----------------------------------------------------------------------------------------------------------------

% Hier die persönlichen Daten eingeben:

\newcommand{\titel}{Titel der Praxis- / Studien- / Haus- oder Bachelor-Arbeit}
\newcommand{\untertitel}{ggf. Untertitel mit ergänzenden Hinweisen}
\newcommand{\arbeit}{Praxisbericht / Studienarbeit / Hausarbeit / Bachelorarbeit}
\newcommand{\studiengang}{Elektrotechnik}
\newcommand{\studienrichtung}{Fahrzeugelektronik}
\newcommand{\studienschwerpunkt}{}
\newcommand{\autor}{Autor(en)}
\newcommand{\matrikelnr}{123 456}
\newcommand{\kurs}{TFE19-2}
\newcommand{\firma}{Firma (Angabe entfällt ggf. bei Studienarbeit)}
\newcommand{\abgabe}{\today}
\newcommand{\betreuerdhbw}{Gutachter der DHBW (nur bei Bachelorarbeit erforderlich)}
\newcommand{\betreuerfirma}{Gutachter der Firma (Angabe entfällt ggf. bei Studienarbeit)}
\newcommand{\jahr}{2019}			% für Angabe im Copyright-Vermerk der Titelseite

% Folgende Zeilen definieren Abkürzungen, um Befehle schneller eingeben zu können
\newcommand{\ua}{\mbox{u.\,a.\ }}
\newcommand{\zB}{\mbox{z.\,B.\ }}
\newcommand{\bs}{$\backslash$}

% Folgende Zeilen weden benötigt, um Tikz und PGF-Plot-Grafiken einzubinden
\usepackage{pgfplots}
\usepackage{pgfplotstable}
\pgfplotsset{compat=newest,width=0.6\linewidth}
\usepgfplotslibrary{smithchart}
\usepackage{tikz}						% Tikz sollte nach Listings Pakete geladen werden.
\usetikzlibrary{arrows}

\hyphenation{Schrift-ar-ten}


% -------------------------------------------------------------------------------------------
%                     Beginn des Dokumenteninhalts
% -------------------------------------------------------------------------------------------
\begin{document}
\let\texteuro\euro						% Eingabe \texteuro, € oder \euro erzeugt gleiches Ergebnis
\setcounter{secnumdepth}{3}				% Nummerierungstiefe fürs Inhaltsverzeichnis
\setcounter{tocdepth}{3}
\sffamily								% für die Titelei serifenlose Schrift verwenden

% ------------------------------ Titelei -----------------------------------------------------

\include{pages/titelseite} 				% erzeugt die Titelseite
\pagenumbering{roman}					% kleine, römische Seitenzahlen für Titelei
\include{pages/erklaerung} 				% Einbinden der eidestattlichen Erklärung
\include{chapter/abstract}   			% Einbinden des Abstracts

\tableofcontents						% Erzeugen des Inhalsverzeichnisses
\cleardoublepage

% --------------------------------------------------------------------------------------------
%                    Inhalt der Bachelorarbeit
%---------------------------------------------------------------------------------------------
\pagenumbering{arabic}					% arabische Seitenzahlen für den Hauptteil

\rmfamily

\include{chapter/einleitung}
\include{chapter/grundlagen}
\include{chapter/konzeptentwurf}
\include{chapter/umsetzung}
\include{chapter/verifikation}
\include{chapter/zusammenfassung}

% ---- Literaturverzeichnis ----------
\interlinepenalty 10000					% Verhindert einen Umbruch mitten in Literatureinträgen
\printbibliography						% Erstellen des Literaturverzeichnisses

% -----Ausgabe aller Verzeichnisse ---
\setlength{\parskip}{0.5\baselineskip}
\renewcommand{\indexname}{Sachwortverzeichnis}
\printindex								% Erzeugen des Indexverzeichnises
\addcontentsline{toc}{chapter}{\indexname}
\input{pages/abkuerzungen}				% Datei mit allgemeinen Abkürzungen laden
\renewcommand{\nomname}{Verzeichnis verwendeter Formelzeichen und Abkürzungen}
\setlength{\nomlabelwidth}{.20\hsize}
\renewcommand{\nomlabel}[1]{#1 \dotfill}
\setlength{\nomitemsep}{-\parsep}
\printnomenclature						% Erzeugen des Abkürzungsverzeichnises, siehe auch Inhalt der Datei pages/abkuerzungen.tex
\cleardoublepage
%\renewcommand{\glossaryname}{Glossar}
%\printglossaries
%\cleardoublepage
\listoffigures 							% Erzeugen des Abbildungsverzeichnisses 
\cleardoublepage
\listoftables 							% Erzeugen des Tabellenverzeichnisses
\cleardoublepage

% -----Anhang ------------------------

\appendix
\clearpage
%\pagenumbering{Roman}					% große, römische Seitenzahlen für Anhang, falls gewünscht
\include{chapter/anhang}
\addchap{Anhang E}
\setcounter{chapter}{5}
\setcounter{section}{0}
\setcounter{table}{0}
\setcounter{figure}{0}

\section{Wichtige \LaTeX -Befehle}

\begin{tabbing}
\hspace*{0cm} \= \hspace{0.28\linewidth} \= \+\kill
\textbackslash \textit{label}\{\}	\> Definition eines Labels, auf welches referenziert werden kann\\ 
	\> z.B.: \textbackslash \textit{label}\{fig:MyImage\}\\ 
\textbackslash \textit{ref}\{\}	\> Setzen einer Referenz zu einem Label\\
\textbackslash \textit{pageref}\{\}	\> Gibt die Seitenzahl zu einer Referenz zurück\\
	\> z.B.: Tabelle\~{}\textbackslash \textit{ref}\{tab:messdaten\} fasst die Messergebnisse zusammen.\\ 
\textbackslash \textit{cite}\{\}	\> Literaturreferenz einfügen\\
\textbackslash \textit{cite}[S. x]\{\}	\> Literaturreferenz mit Angabe einer Seitenzahl \glqq x\grqq~einfügen\\

\textbackslash \textit{footnote}\{\}	\> Fußnote einfügen\\ 
\~{}	\> Einfügen eines geschützten Leerzeichens\\ 
\textdollar \textit{Formel} \textdollar	\> Eingabe einer Formel im Text\\
\textbackslash \textit{nomenclature}\{a.\}\{ab\}	\> Aufnahme der Abkürzung \glqq a.\grqq~für \glqq ab\grqq~in das Abkürzungsverzeichnis.\\
\textbackslash \textit{index}\{Obst!Birne\} \> Aufnahme des Begriffs \glqq Birne\grqq~in den Index unter \glqq Obst\grqq. \index{Obst!Birne} \\
\textbackslash \textit{clearpage}	\> Ausgabe aller Gleitobjekte und Umbruch auf neue Seite\\ 
\end{tabbing}

\clearpage

\section{Vorlagen für \LaTeX Umgebungen}

\subsection{Listen und Aufzählungen}

Es gibt folgende Listentypen. Die wichtigsten:

\begin{itemize}
	\item Einfache Liste mit \textit{itemize}-Umgebung
	\item ...
\end{itemize}

\begin{enumerate}
	\item Nummerierte Liste mit \textit{enumerate}-Umgebung
	\item ...
\end{enumerate}

\begin{enumerate}[label=\alph*.]
	\item wobei man bei der \textit{enumerate}-Umgebung leicht die Art der Nummerierung ändern kann,
	\item ...
\end{enumerate}

und durch verschachtelte Umgebungen verschiedene Aufzählungsebenen darstellen kann:

\begin{enumerate}[label=\alph*)]
	\item Erster Aufzählungspunkt der ersten Ebene
	\item ...
	\begin{itemize}
		\item Erster Punkt der zweiten Ebene
		\item Zweiter Punkt der zweiten Ebene
	\end{itemize}
	\item Das sollte an Beispielen zunächst einmal genügen.
\end{enumerate}

\clearpage

\subsection{Bilder und Grafiken}

Bilder können als PDF-, JPG-, und PNG-Bilder in \LaTeX eingebunden werden. Damit eine Grafik in hoher Qualität dargestellt wird, sollte das Dateiformat der Grafik vektorbasiert sein, d.h. als PDF-Datei vorliegen. Viele Zeichenprogramme unterstützen einen PDF-Export (z.B. GIMP, Adobe Illustrator, etc.). Für Grafiken aus PowerPoint sei folgende Vorgehensweise beim Export empfohlen:

\begin{enumerate}
	\item Die gewünschte Grafik in PowerPoint zeichnen.
	\item Gewünschten Bildbereich markieren, rechte Maustaste klicken und \glqq Als Grafik speichern ...\grqq~wählen.
	\item Grafik im Format EMF abspeichern. Das EMF-Format ist vektorbasiert.\footnote{Mit dem Mac kann in PowerPoint die Grafik direkt im PDF-Format exportiert werden. Die weiteren Schritte entfallen daher.}
	\item Mit dem Programm XnView die Grafik im EMF-Format in PDF wandeln und abspeichern.
	\item Die so erzeugte PDF-Datei enthält eine vektorbasierte Grafik und kann in \LaTeX~ eingebunden werden.
\end{enumerate}

Abbildung~\ref{fig:MyImage} zeigt ein Beispielbild einer Grafik, welche aus PowerPoint exportiert wurde.

\begin{figure}[hbt]
	\centering
	\includegraphics[width=0.3\linewidth]{images/MyImage}
	\caption[Beispiel für die Einbindung eines Bildes.]{Beispiel für die Einbindung eines Bildes (PDF-, JPG-, und PNG-Bilder können eingebunden werden).}
	\label{fig:MyImage}
\end{figure}

Der Quellcode des Beispielbildes aus Abbildung~\ref{fig:MyImage} ist in Listing~\ref{lst:fig} zu sehen.

\clearpage

\begin{lstlisting}[caption=Quellcode der Abbildung~\ref{fig:MyImage}.,label=lst:fig]
\begin{figure}[hbt]				% here, bottom, top
\centering						% Zentrierung
\includegraphics[width=0.6\linewidth]{images/MyImage}		
\caption[Beispiel für die Einbindung eines Bildes.]{Beispiel für die Einbindung eines Bildes (PDF-, JPG-, und PNG-Bilder können eingebunden werden).}
\label{fig:MyImage}
\end{figure}
\end{lstlisting}

Grafiken können auch mithilfe des Packages Tikz gezeichnet, bzw. programmiert werden. Grafiken mit Tikz werden mit dem \textit{input}-Befehl in die \textit{figure}-Umgebung geladen, wie nachfolgendes Beispiel in Abbildung~\ref{fig:tikz_house} zeigt:

\begin{figure}[hbt]
	\centering
	\input{tikz/tikz_house.tex}
	\caption[Mit Tikz programmierte Grafik.]{Mit Tikz programmierte Grafik.}
	\label{fig:tikz_house}
\end{figure}

Ein etwas umfangreicheres Beispiel zur Digitaltechnik ist in Abbildung~\ref{fig:tikz_digital} dargestellt:

\begin{figure}[hbt]
	\centering
	\input{tikz/tikz_digital.tex}
	\caption[Mit Tikz programmierte Grafik, welche bereits vorgefertigte Bibliotheken für Symbole aus der Digitaltechnik nutzt.]{Mit Tikz programmierte Grafik, welche bereits vorgefertigte Bibliotheken für Symbole aus der Digitaltechnik nutzt.}
	\label{fig:tikz_digital}
\end{figure}

\clearpage

In der Tikz-Umgebung können auch Diagramme mit dem \textit{pgfplot}-Befehlssatz erzeugt werden. In Abbildung \ref{fig:pgfplot} sehen Sie ein Beispiel.

\begin{figure}[hbt]
	\centering
	\input{pgfplot/mess_fehlerbalken.tex}
	\caption[Diagramm, erstellt mit dem \textit{pgfplot}-Befehlssatz.]{Ein Diagramm, erstellt in der \textit{tikzpicture}-Umgebung mit dem \textit{pgfplot}-Befehlssatz. Das Diagramm stellt Messdaten, deren Fehlerbalken und eine Regressionskurve dar. Die Messdaten werden von einer separaten Datei eingelesen und die Regressionskurve wurde mit \textit{pgfplot} berechnet und erstellt.}
	\label{fig:pgfplot}
\end{figure}

\clearpage

Auch hierzu der Quellcode in Listing~\ref{lst:pgfplot}.

\begin{lstlisting}[caption=Quellcode der Abbildung~\ref{fig:pgfplot}.,label=lst:pgfplot]
\begin{figure}[hbt]
\centering
\input{pgfplot/mess_fehlerbalken.tex}
\caption[Diagramm, erstellt mit dem \textit{pgfplot}-Befehlssatz.]{Ein Diagramm, erstellt in der \textit{tikzpicture}-Umgebung mit dem \textit{pgfplot}-Befehlssatz. Das Diagramm stellt Messdaten, deren Fehlerbalken und eine Regressionskurve dar. Die Messdaten werden von einer separaten Datei eingelesen und die Regressionskurve wurde mit \textit{pgfplot} berechnet und erstellt.}
\label{fig:pgfplot}
\end{figure}
\end{lstlisting}

In Listing~\ref{lst:tikz} ist der Quellcode der Datei \textit{mess\_fehlerbalken.tex} dargestellt.

\begin{lstlisting}[caption=Quellcode der Datei \textit{mess\_fehlerbalken.tex}.,label=lst:tikz]
\begin{tikzpicture}
\begin{axis}[scale=1.3,legend entries={Messwerte mit Fehlerbalken,
$\pgfmathprintnumber{\pgfplotstableregressiona} \cdot x  
\pgfmathprintnumber[print sign]{\pgfplotstableregressionb}$}, legend style={draw=none},legend style={at={(0.01,0.98)},anchor=north west},xlabel=Stromstärke $I \; \mathrm{ \lbrack mA \rbrack}$,ylabel=Spannung $U \; \mathrm{ \lbrack V \rbrack}$]
\addlegendimage{mark=*,blue}
\addlegendimage{no markers,red}
\addplot+[error bars/.cd, y dir=both,y explicit]
table[x=x,y=y,y error=errory] 
{pgfplot/messdaten_mitfehler.dat};
\addplot table[mark=none,y={create col/linear regression={y=y}}]
{pgfplot/messdaten_mitfehler.dat};
\end{axis}
\end{tikzpicture}
\end{lstlisting}

\clearpage

In Abbildung~\ref{fig:pgfplot2y} wird ein weiters Beispiel für ein Diagramm gezeigt. Oftmals wird eine zweite y-Achse verwendet, um verschiedene Skalen darstellen zu können.

\begin{figure}[hbt]
	\centering
	\input{pgfplot/mess_zweiyachsen.tex}
	\caption[Diagramm mit zwei unterschiedlichen y-Achsen.]{Diagramm mit zwei unterschiedlichen y-Achsen.}
	\label{fig:pgfplot2y}
\end{figure}

\clearpage

\subsection{Tabellen}

\begin{table}[hbt]	
	\centering
	\renewcommand{\arraystretch}{1.5}	% Skaliert die Zeilenhöhe der Tabelle
	\captionabove[Liste der verwendeten Messgeräte]{Liste der verwendeten Messgeräte. Die Genauigkeitsangaben beziehen sich auf die Standardabweichung $1\cdot \sigma$.}
	\label{tab:bsp}
	\begin{tabular}{ccccc}
		\textbf{Messgerät} & \textbf{Hersteller} & \textbf{Typ} & \textbf{Verwendung} & \textbf{Genauigkeit}\\ 
		\hline 
		\hline 
		\parbox[t]{0.2\linewidth}{\centering Spannungs-\\versorgung} & Voltmaker & HV2000 & \parbox[t]{0.2\linewidth}{\centering Spannungs-\\versorgung der\\Platine} & $\Delta U = \pm 5 $~mV \\ % Der parbox-Befehl ist erforderlich, damit ein Zeilenumbruch erzeugt werden kann. c-Spalten (zentriert) erlauben nicht automatisch einen Zeilenumpruch. Linksbündig gesetzte p-Spalten erlauben automatisch den Zeilenumbruch.
		Strommessgerät & Currentcount & Hotamp 16 & \parbox[t]{0.2\linewidth}{ \centering Strommessung\\am Versorgungspin\\des µC} & $\Delta I = \pm 0.1$~A \\ 
		\hline 
	\end{tabular} 
\end{table}

Der Quellcode der Beispieltabelle~\ref{tab:bsp} ist in Listing~\ref{lst:tab} zu sehen.

\begin{lstlisting}[caption=Quellcode der Tabelle~\ref{tab:bsp}.,label=lst:tab]
\begin{table}[hbt]	
\centering
\renewcommand{\arraystretch}{1.5}	% Skaliert die Zeilenhöhe der Tabelle
\captionabove[Liste der verwendeten Messgeräte]{Liste der verwendeten Messgeräte. Die Genauigkeitsangaben beziehen sich auf die Standardabweichung $1\cdot \sigma$.}
\label{tab:bsp}
\begin{tabular}{ccccc}
\textbf{Messgerät} & \textbf{Hersteller} & \textbf{Typ} & \textbf{Verwendung} & \textbf{Genauigkeit}\\ 
\hline 
\hline 
\parbox[t]{0.2\linewidth}{\centering Spannungs-\\versorgung} & Voltmaker & HV2000 & \parbox[t]{0.2\linewidth}{\centering Spannungs-\\versorgung der\\Platine} & $\Delta U = \pm 5 $~mV \\ % Der parbox-Befehl ist erforderlich, damit ein Zeilenumbruch erzeugt werden kann. c-Spalten (zentriert) erlauben nicht automatisch einen Zeilenumpruch. Linksbündig gesetzte p-Spalten erlauben automatisch den Zeilenumbruch.
Strommessgerät & Currentcount & Hotamp 16 & \parbox[t]{0.2\linewidth}{ \centering Strommessung\\ am Versorgungspin\\ des \textmu C} & $\Delta I = \pm 0.1$~A \\ 
\hline 
\end{tabular} 
\end{table}
\end{lstlisting}

\clearpage

\subsection{Formeln}

Formeln lassen sich in \LaTeX~ganz einfach schreiben. Es gibt unterschiedliche Umgebungen zum Schreiben von Formeln. Z.B. direkt im Text $v=s/t$ oder abgesetzt

\[F=m \cdot a\]

oder auch, wie in wissenschaftlichen Dokumenten üblich, nummeriert

\begin{equation}
P=\frac{U^2}{R} \quad .
\label{eqn:leistung}
\end{equation}

Mit einem Label in Formel~\ref{eqn:leistung} lassen sich natürlich auch Formeln im Text referenzieren. \LaTeX~verwendet im Formelmodus einen eigenen Schriftsatz, welcher entsprechend der gängigen Konventionen kursive Zeichen verwendet. Sollen im Formelmodus Einheiten in normaler Schriftart eingefügt werden, dann kann dies über den Befehl \textbackslash \textit{mathrm}\{\} erwirkt werden, wie im Quellcode von Formel~\ref{eqn:leistungMitEinh} zu sehen ist.

\begin{equation}
P=\frac{U^2}{R} = \frac{\left( 100~\mathrm{V}\right)^2}{100~\Omega} = 100~\mathrm{W}\quad .
\label{eqn:leistungMitEinh}
\end{equation}

Zum direkten Vergleich sind die Einheiten in Formel~\ref{eqn:leistungMitEinhfalsch} falsch dargestellt:

\begin{equation}
P=\frac{U^2}{R} = \frac{\left( 100~V\right)^2}{100\,\varOmega} = 100\,W
\label{eqn:leistungMitEinhfalsch}
\end{equation}

Zur einfachen Eingabe von Einheiten kann auch das Package \textbackslash \textit{siunitx} verwendet werden:

\begin{equation}
	P=\SI{100}{\watt}=\SI{100}{\joule\per\second}
\end{equation}

Das sind nur ein paar wenige Beispiele und es gibt sehr viele Packages, um Besonderheiten in Formeln realisieren zu können, z.B. mehrzeilige Formeln mit vertikaler Ausrichtung. Nennen Sie Formeln nur, wenn diese zum besseren Verständnis auch wirklich nützlich sind.

Folgende Befehle sind innerhalb von Formel-Umgebungen nützlich:
\begin{tabbing}
	\hspace*{0cm} \= \hspace{0.35\linewidth} \= \+\kill
	\textbackslash \textit{text}\{\}	\> Damit kann in Formel-Umgebung Text geschrieben werden.\\ 
	\textbackslash, \textbackslash: \textbackslash; oder \textbackslash quad und \textbackslash qquad \> Zusätzlichen Abstand zwischen Symbolen einfügen.\\
	\textbackslash \textit{notag} \> Nummerierung einer bestimmten Formel ausschalten.
\end{tabbing}

Abschließend nochmals ein kleines Beispiel:

\begin{eqnarray}
\sum\limits_{n=1}^\infty f\left(x_n\right)\cdot \Delta x=  \lim\limits_{\Delta x \rightarrow 0} \frac{f\left(x_0+\Delta x\right)-f\left(x_0\right)}{\Delta x} = \frac{\diff f}{\diff x} = \dot{f}(x)
\end{eqnarray}		% Zeile auskommentieren bei finalem Dokument!

\end{document}